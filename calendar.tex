\documentclass{article}
\usepackage{standalone}
\usepackage{import}

\import{./}{preamble}

\begin{document}

\section{Course Calendar (one topic per class meeting)}

\subsection{Facilitation, Safe Spaces, and Ouch/Oops}
\paragraph{Goals} This session is devoted to allowing you to introduce yourself and share your personal experiences with the topics that we will explore during this course.  In particular, we will discuss how race and gender have played a role in your participation in the sciences.  In order to create a safe space for the sharing of personal narratives the readings cover facilitation skills.

\paragraph{Questions}

\paragraph{Readings}
\begin{itemize}
\item Barbara Mae Gayle, Derek Cortez, and Raymond W. Preiss. Safe Spaces, Difficult Dialogues, and Critical Thinking. International Journal for the Scholarship of Teaching and Learning 7.2 (2013): 1-8. Web: \url{http://digitalcommons.georgiasouthern.edu/ij-sotl/vol7/iss2/5/}.
\item Leslie C. Aguilar. Ouch! That Stereotype Hurts... Communicating Respectfully in a Diverse World. Bedford, TX: The Walk The Talk Company, 2006. Adapted in The Ouch! Files. Vol. 5, No. 1. Jan 2014. Web: \url{http://www.ouchthatstereotypehurts.com/OUCH_Files/Archives/Ouch_Vol5No1.html}.
\item A Community Builder's Tool Kit: A Project of The Institute for Democratic Renewal and Project Change Anti-Racism Initiative. Web: \url{http://www.capd.org/pubfiles/pub-2004-07-03.pdf}.
\end{itemize}


\subsection{Philosophy of Science and the Myth of Objectivity}
\begin{itemize}
\item Thomas S. Kuhn. The Structure of Scientific Revolutions: 50th Anniversary Edition. University of Chicago Press. 2012. Ch. 2-3, The Route to Normal Science, The Nature of Normal Science, pp. 10-34. Ch. 9, The Nature and Necessity of Scientific Revolutions, pp. 92-110.
\item Helen E. Longino. Science as Social Knowledge: Values and Objectivity in Scientific Inquiry. Princeton University Press. 1990. Ch. 4, Values and Objectivity, pp. 62-82.
\item Sir Francis Bacon. The New Atlantis. 1627.
\end{itemize}

\subsection{A Feminist History of Science}
\begin{itemize}
\item Sandra Harding. The Science Question in Feminism. pp. 15-49, 136-162. 1986. Web: \url{https://www.andrew.cmu.edu/course/76-327A/readings/Harding.pdf}.
\item Donna Haraway. Situated Knowledges: The Science Question in Feminism and the Privilege of Partial Perspective. Feminist Studies, Vol. 14, No. 3. (1988), pp. 575-599. Web: \url{http://doi.org/10.2307/3178066}.
\end{itemize}

\subsection{Background and Statistics}
\begin{itemize}
\item National Science Foundation, National Center for Science and Engineering Statistics. 2015. Women, Minorities, and Persons with Disabilities in Science and Engineering: 2015. Special Report NSF 15-311. Arlington, VA. Web: \url{http://www.nsf.gov/statistics/wmpd/} (read the digest, skim the full report).
\end{itemize}

\subsection{Gender}
\begin{itemize}
\item Beyond Bias and Barriers. National Academy of Sciences Report. Web: \url{http://www.med.upenn.edu/focus/user_documents/bias_summary.pdf}.
\end{itemize}

\subsection{Underrepresented Minorities}
\begin{itemize}
\item Peggy McIntosh. White Privilege: Unpacking the Invisible Backpack. Web: \url{https://www.isr.umich.edu/home/diversity/resources/white-privilege.pdf}.
\item Howard Garrison. Underrepresentation by Race-Ethnicity across Stages of U.S. Science and Engineering Education. CBE Life Sciences Education. Vol 12, pp. 357-363. 2013. Web: \url{http://doi.org/10.1187/cbe.12-12-0207}.
\item Joan C. Williams, Katherine W. Phillips, Erika V. Hall. Double Jeopardy? Gender Bias Against Women of Color in Science. UC Hastings College of Law. 2014. Web: \url{http://www.uchastings.edu/news/articles/2015/01/double-jeopardy-report.pdf}
\end{itemize}


\subsection{Scientists and Identity }
\begin{itemize}
\item Johnson, A., Brown, J., Carlone, H., Cuevas, A. K. (2011). Authoring identity amidst the treacherous terrain of science: A multiracial feminist examination of the  journeys of three women of color in science. Journal of Research in Science Teaching, 48(4).  pp. 339-366. Web: \url{http://doi.org/10.1002/tea.20411}.
\item Brian A. Nosek et al. National differences in gender-science stereotypes predict national sex differences in science and math achievement. Proceedings of the National Academy of Sciences of the United States 106.26. pp. 10593-10597. 2009. Web: \url{http://doi.org/10.1073/pnas.0809921106}.
\end{itemize}

\subsection{Stereotype Threat and Imposter Syndrome}
\paragraph{Possible guest facilitators:} Prof. Anne Charity Hudley, Prof. Cheryl Dickter
\begin{itemize}
\item Claude M. Steele. Whistling Vivaldi: How Stereotypes Affect Us and What We Can Do. Norton \& Company, Inc. 2010. Chapters 1-2, 9-10.
\end{itemize}

\subsection{Science Education}
\paragraph{Possible guest facilitators:} Prof. Rio Riofrio
\begin{itemize}
\item Paulo Freire. Pedagogy of the Oppressed.
\end{itemize}

\subsection{Science Communication}

\subsection{Policy}

\subsection{Change}
\begin{itemize}
\item Ong, M. (2005). Body Projects of Young Women of Color in Physics: Intersections of Gender, Race, and Science. Social Problems, 52(4), 593J617. Web: \url{http://doi.org/10.1525/sp.2005.52.4.593}.
\end{itemize}

\subsection*{Other Resources}
\begin{itemize}
\item Cech, E., Waidzunas, T. (2011). Navigating the Heteronormativity of Engineering: The  experiences of lesbian, gay, and bisexual students. Engineering Studies, 3(1), 1J24. Web: \url{http://doi.org/10.1080/19378629.2010.545065}.
\item Ceci, S. J., Williams, W. M. (2010). Understanding current causes of women's  underrepresentation in science. Nature Neuroscience, 2010, 1J6. Web: \url{http://doi.org/10.1073/pnas.1014871108}.
\end{itemize}

\end{document}
