\documentclass{article}
\usepackage{standalone}
\usepackage{import}

\import{./}{preamble}

\begin{document}

\section{Course Calendar}


\subsection{Week 1 - Introduction and Logistics}
This session is devoted to the introduction of the participants and of the topics that we will explore during this course. In this course we will discuss how race and gender have played a role in the unequal opportunities for participation in the sciences. In order to create a safe space for the sharing of personal narratives, the first readings will cover facilitation skills.

\subsection{Week 2 - Facilitation, Safe Spaces, and Ouch/Oops}
\paragraph{Goals} 
\paragraph{Readings}
\begin{itemize}
\item Barbara Mae Gayle, Derek Cortez, and Raymond W. Preiss. Safe Spaces, Difficult Dialogues, and Critical Thinking. International Journal for the Scholarship of Teaching and Learning 7.2 (2013): 1-8. Web: \url{http://digitalcommons.georgiasouthern.edu/ij-sotl/vol7/iss2/5/}. (8 pages)
\item Leslie C. Aguilar. Ouch! That Stereotype Hurts... Communicating Respectfully in a Diverse World. Bedford, TX: The Walk The Talk Company, 2006. Adapted in The Ouch! Files. Vol. 5, No. 1. Jan 2014. Web: \url{http://www.ouchthatstereotypehurts.com/OUCH_Files/Archives/Ouch_Vol5No1.html}. (1 page)
\item A Community Builder's Tool Kit: A Project of The Institute for Democratic Renewal and Project Change Anti-Racism Initiative. Web: \url{http://www.capd.org/pubfiles/pub-2004-07-03.pdf}. (37 pages)
\end{itemize}
\paragraph{Question prompts}
\begin{itemize}
\item Can you identify in your personal experience a situation where the frameworks described in the readings would have been applicable?
\item What is your familiarity with the facilitation frameworks described in the readings? They could be either formally or informally, fully or partially applied in groups that you are part of.
\end{itemize}


\subsection{Week 3 - Philosophy of Science and the Myth of Objectivity}
\paragraph{Readings}
\begin{itemize}
\item Thomas S. Kuhn. The Structure of Scientific Revolutions: 50th Anniversary Edition. University of Chicago Press. 2012. Ch. 2-3, The Route to Normal Science, The Nature of Normal Science, pp. 10-34. Ch. 9, The Nature and Necessity of Scientific Revolutions, pp. 92-110. Web: \url{http://projektintegracija.pravo.hr/_download/repository/Kuhn_Structure_of_Scientific_Revolutions.pdf}. (24 pages + 18 pages)
\item Helen E. Longino. Science as Social Knowledge: Values and Objectivity in Scientific Inquiry. Princeton University Press. 1990. Ch. 4, Values and Objectivity, pp. 62-82. Web: \url{http://joelvelasco.net/teaching/3330/longino-valuesandobjectivity.pdf}. (21 pages)
\item Sir Francis Bacon. The New Atlantis. 1627. Web: \url{http://www.fcsh.unl.pt/docentes/rmonteiro/pdf/The_New_Atlantis.pdf} (optional)
\end{itemize}
\paragraph{Question prompts}


\subsection{Week 4 - A Feminist History of Science}
\paragraph{Readings}
\begin{itemize}
\item Sandra Harding. The Science Question in Feminism. pp. 15-49, 136-162. 1986. Web: \url{https://www.andrew.cmu.edu/course/76-327A/readings/Harding.pdf}.
\item Donna Haraway. Situated Knowledges: The Science Question in Feminism and the Privilege of Partial Perspective. Feminist Studies, Vol. 14, No. 3. (1988), pp. 575-599. Web: \url{http://doi.org/10.2307/3178066}.
\end{itemize}
\paragraph{Question prompts}


\subsection{Week 5 - Background and Statistics}
\paragraph{Readings}
\begin{itemize}
\item National Science Foundation, National Center for Science and Engineering Statistics. 2015. Women, Minorities, and Persons with Disabilities in Science and Engineering: 2015. Special Report NSF 15-311. Arlington, VA. Web: \url{http://www.nsf.gov/statistics/wmpd/} (read the digest, skim the full report).
\end{itemize}
\paragraph{Question prompts}


\subsection{Week 6 - Gender}
\paragraph{Readings}
\begin{itemize}
\item Beyond Bias and Barriers. National Academy of Sciences Report. Web: \url{http://www.med.upenn.edu/focus/user_documents/bias_summary.pdf}.
\end{itemize}
\paragraph{Question prompts}


\subsection{Week 7 - Underrepresented Minorities}
\paragraph{Readings}
\begin{itemize}
\item Peggy McIntosh. White Privilege: Unpacking the Invisible Backpack. Web: \url{https://www.isr.umich.edu/home/diversity/resources/white-privilege.pdf}.
\item Howard Garrison. Underrepresentation by Race-Ethnicity across Stages of U.S. Science and Engineering Education. CBE Life Sciences Education. Vol 12, pp. 357-363. 2013. Web: \url{http://doi.org/10.1187/cbe.12-12-0207}.
\item Joan C. Williams, Katherine W. Phillips, Erika V. Hall. Double Jeopardy? Gender Bias Against Women of Color in Science. UC Hastings College of Law. 2014. Web: \url{http://www.uchastings.edu/news/articles/2015/01/double-jeopardy-report.pdf}.
\end{itemize}
\paragraph{Question prompts}

\subsection{Week 8 - LGBT+ Experiences in STEM}
\paragraph{Readings}
\begin{itemize}
\item Cech, E., Waidzunas, T. (2011). Navigating the Heteronormativity of Engineering: The  experiences of lesbian, gay, and bisexual students. Engineering Studies, 3(1), 1J24. Web: \url{http://doi.org/10.1080/19378629.2010.545065}.
\item LGBT+ Physicists. (2015). Supporting LGBT+ Physicists \& Astronomers: Best Practices for Academic Departments. Web: \url{http://lgbtphysicists.org/files/BestPracticesGuide.pdf}.
\end{itemize}
\paragraph{Question prompts}

\subsection{Week 9 - Scientists and Identity}
\paragraph{Readings}
\begin{itemize}
\item Johnson, A., Brown, J., Carlone, H., Cuevas, A. K. (2011). Authoring identity amidst the treacherous terrain of science: A multiracial feminist examination of the  journeys of three women of color in science. Journal of Research in Science Teaching, 48(4).  pp. 339-366. Web: \url{http://doi.org/10.1002/tea.20411}.
\item Brian A. Nosek et al. National differences in gender-science stereotypes predict national sex differences in science and math achievement. Proceedings of the National Academy of Sciences of the United States 106.26. pp. 10593-10597. 2009. Web: \url{http://doi.org/10.1073/pnas.0809921106}.
\end{itemize}
\paragraph{Question prompts}


\subsection{Week 10 - Stereotype Threat and Imposter Syndrome}
%Possible guest facilitators: Prof. Anne Charity Hudley, Prof. Cheryl Dickter
\paragraph{Readings}
\begin{itemize}
\item Claude M. Steele. Whistling Vivaldi: How Stereotypes Affect Us and What We Can Do. Norton \& Company, Inc. 2010. Chapters 1-2, 9-10.
\item Ed Yong. 15-minute writing exercise closes the gender gap in university-level physics. Discover Magazine, November 25, 2010. Web: \url{http://blogs.discovermagazine.com/notrocketscience/2010/11/25/15-minute-writing-exercise-closes-the-gender-gap-in-university-level-physics/}. (2 pages)
\item Akira Miyake et al. Reducing the Gender Achievement Gap in College Science: A Classroom Study of Values Affirmation. Science, Vol. 330, no. 6008, pp. 1234-1237. 2010. Web: \url{http://www.sciencemag.org/content/330/6008/1234}. (4 pages)
\end{itemize}


\subsection{Week 11 - Change}
\paragraph{Readings}
\begin{itemize}
\item Ong, Maria (2005). Body Projects of Young Women of Color in Physics: Intersections of Gender, Race, and Science. Social Problems, 52(4), 593J617. Web: \url{http://doi.org/10.1525/sp.2005.52.4.593}. (25 pages)
\end{itemize}
\paragraph{Question prompts}


\subsection{Week 12-13 - Projects}
These weeks will be used for project discussions.


\subsection{Week 15 - Presentations}
During the last week of classes you will present your project or final paper.


\subsection*{Other Resources}
\begin{itemize}
\item Paulo Freire. Pedagogy of the Oppressed. Web: \url{https://libcom.org/files/FreirePedagogyoftheOppressed.pdf}.
\item Ceci, S. J., Williams, W. M. (2010). Understanding current causes of women's  underrepresentation in science. Nature Neuroscience, 2010, 1J6. Web: \url{http://doi.org/10.1073/pnas.1014871108}.
\end{itemize}

\end{document}
