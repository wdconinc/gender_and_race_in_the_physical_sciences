\documentclass{article}
\usepackage{standalone}
\usepackage{import}

\import{./}{preamble}

\begin{document}

\section{Instructors}
\begin{itemize}
\item Instructor: Wouter Deconinck (Physics), wdeconinck@wm.edu
\item Possible co-instructors and guest moderators: Prof. Anne Charity Hudley, Prof. Jennifer Putzi, Prof. Rio Riofrio, Prof. Leisa Meyer, your name here
\end{itemize}

\section{Course description}
Science purports to be the systematic study of the physical and natural world through observation and experimentation. This empirical approach supposedly does not depend on the scientist’s gender, race, ethnicity, or nationality. Why, then, does being a scientist depend on all these things? In this course we will explore this question through interdisciplinary readings in history and philosophy of science, sociology and social psychology, and current activism resources. The class sessions will be strongly discussion-oriented. In addition to participation in discussions, a semester-long project will serve as the form of final evaluation of engagement with the course material.

\section{Course Organization}
This 1-credit course will meet once per week for 12 weeks. The second half of the semester will consist of a course project (alone or in small groups of up to two students) which could consist of continued literature review, an on-campus activism project, or other. The projects will be presented at a public event during the last week of classes. A final report will be due by the end of the final exams period. 

\section{Course Topics}
\begin{itemize}
\item Statistics of representation
\item Women in science
\item Underrepresented minorities
\item Well-represented minorities
\item Stereotype threat
\item Science education
\item Historical perspectives
\item Women scientists today
\item LGBT+ scientists
\end{itemize}

\end{document}
